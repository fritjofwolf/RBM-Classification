% Gibt die Klasse des Dokuments an
\documentclass[a4paper]{scrartcl}
% passende Kodierung für deutsche Sonderzeichen
\usepackage[T1]{fontenc}
% bequemer Eingabezeichensatz(für Eszett etc.)
\usepackage[utf8]{inputenc}
% Für align-Umgebung etc.
\usepackage{amsmath, amsthm}


% deutsche Silbentrennung/Rechtschreibung
\usepackage[ngerman]{babel}
\usepackage{amssymb}


% Eigene Befehle
\newtheorem{thm}{Theorem}[section]
\newtheorem{cor}[thm]{Korollar}
\newtheorem{defi}[thm]{Definition}
\newtheorem{lem}[thm]{Lemma}
\newtheorem{bem}[thm]{Bemerkung}

% Standardmengen
\newcommand{\R}{\mathbb{R}}
\newcommand{\Q}{\mathbb{Q}}
\newcommand{\Z}{\mathbb{Z}}
\newcommand{\N}{\mathbb{N}}
\newcommand{\E}{\mathbb{E}}
\newcommand{\rainf}{\rightarrow \infty}



\begin{document}
% Überschrift
\title{Classification using Restricted Boltzmann Machines}
\author{Katarzyna Tarnowska \and Fritjof Wolf}
\maketitle
\newpage
% Abstract
\begin{abstract}
\textbf{Abstract:} 
\end{abstract}

% Textbody
\section{Introduction}


\section{Empirical Results for Discriminative RBM}
\subsection{A}
\subsection{B}

% Literatur
\begin{thebibliography}{9}
   \bibitem[1]{Hopfield} Hopfield, J.J. (1984) \emph{Neurons with graded response have collective computational properties like those of two-state neurons} (Proc. Natl. Acad. Sci. USA)
   
\end{thebibliography}
\end{document}